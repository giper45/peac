\section{PEaC Integration with External Applications}

This section describes how PEaC can be integrated into external applications through two primary approaches: as a Python library and via Command Line Interface (CLI). The integration patterns enable developers to incorporate structured prompt engineering capabilities into their existing workflows and applications.

\subsection{Integration Approaches}

PEaC supports multiple integration scenarios to accommodate different development environments and use cases. Table~\ref{tab:integration-approaches} summarizes the available integration methods and their characteristics.

\begin{table}[htbp]
\centering
\caption{PEaC Integration Approaches}
\label{tab:integration-approaches}
\begin{tabular}{|l|p{4cm}|p{3cm}|p{3cm}|}
\hline
\textbf{Method} & \textbf{Description} & \textbf{Use Case} & \textbf{Advantages} \\
\hline
Python Library & Direct import of PEaC modules & Python applications & Fine-grained control, programmatic access \\
\hline
CLI Interface & Command-line execution & Any programming language & Language agnostic, simple integration \\
\hline
Batch Processing & Automated processing of multiple documents & Large-scale operations & Efficient bulk processing \\
\hline
\end{tabular}
\end{table}

\subsection{Python Library Integration}

The Python library integration provides the most flexible approach for incorporating PEaC functionality into existing Python applications. Developers can directly import the \texttt{PromptYaml} class from the \texttt{peac.core} module and programmatically generate prompts.

\subsubsection{Implementation Steps}

The integration process follows these key steps:

\begin{enumerate}
    \item \textbf{Configuration Loading}: The application loads a YAML configuration file containing the prompt engineering specifications
    \item \textbf{PromptYaml Instantiation}: Create a \texttt{PromptYaml} object with the configuration file path
    \item \textbf{Content Processing}: The library processes various data sources (local files, web content, RAG queries) as specified in the configuration
    \item \textbf{Prompt Generation}: Call \texttt{get\_prompt\_sentence()} to generate the structured prompt
    \item \textbf{LLM Integration}: Send the generated prompt to the target LLM service
\end{enumerate}

\subsubsection{Code Example}

\begin{verbatim}
from peac.core.peac import PromptYaml

# Load PEaC configuration
processor = PromptYaml("config.yaml")

# Generate structured prompt
prompt = processor.get_prompt_sentence()

# Send to LLM service
response = llm_service.query(prompt)
\end{verbatim}

\subsection{CLI Integration}

The Command Line Interface provides a language-agnostic integration method, allowing applications written in any programming language to leverage PEaC capabilities through system calls.

\subsubsection{CLI Usage Pattern}

Applications can execute PEaC commands and capture the output through standard input/output mechanisms:

\begin{verbatim}
peac process config.yaml --output prompt.txt
\end{verbatim}

This approach is particularly suitable for:
\begin{itemize}
    \item Applications written in languages other than Python
    \item Microservice architectures where PEaC runs as a separate service
    \item Batch processing scenarios requiring minimal integration overhead
\end{itemize}

\subsection{Batch Processing Integration}

For large-scale operations, PEaC supports batch processing integration where multiple documents or configurations can be processed automatically. This pattern is especially useful for:

\begin{itemize}
    \item Document analysis workflows
    \item Content generation pipelines
    \item Automated report generation systems
\end{itemize}

The batch processing follows an iterative pattern where each input document is processed through the complete PEaC pipeline, generating customized prompts for subsequent LLM processing.

\subsection{Integration Benefits}

The integration capabilities of PEaC provide several advantages for external applications:

\begin{description}
    \item[Modularity] Applications can leverage PEaC's prompt engineering capabilities without reimplementing complex logic
    \item[Consistency] Standardized prompt generation ensures consistent LLM interactions across different parts of an application
    \item[Maintainability] Prompt engineering logic is externalized to configuration files, enabling updates without code changes
    \item[Scalability] Both library and CLI integrations support scaling from single requests to batch operations
\end{description}

\subsection{Error Handling and Validation}

PEaC integration includes comprehensive error handling mechanisms:

\begin{itemize}
    \item Configuration validation ensures YAML files conform to expected schemas
    \item File system access errors are properly caught and reported
    \item Network connectivity issues for web-based content sources are handled gracefully
    \item RAG query failures include fallback mechanisms when knowledge bases are unavailable
\end{itemize}

This robust error handling ensures that external applications can implement reliable prompt generation workflows with appropriate fallback strategies when individual components encounter issues.